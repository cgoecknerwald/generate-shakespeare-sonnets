\newif\ifshowsolutions
\showsolutionstrue
\documentclass{article}
\usepackage{listings}
\usepackage{amsmath}
%\usepackage{subfigure}
\usepackage{subfig}
\usepackage{amsthm}
\usepackage{amsmath}
\usepackage{amssymb}
\usepackage{graphicx}
\usepackage{mdwlist}
\usepackage[colorlinks=true]{hyperref}
\usepackage{geometry}
\usepackage{titlesec}
\geometry{margin=1in}
\geometry{headheight=2in}
\geometry{top=2in}
\usepackage{palatino}
\usepackage{mathrsfs}
\usepackage{fancyhdr}
\usepackage{paralist}
\usepackage{todonotes}
\setlength{\marginparwidth}{2.15cm}
\usepackage{tikz}
\usetikzlibrary{positioning,shapes,backgrounds}
\usepackage{float} % Place figures where you ACTUALLY want it
\usepackage{comment} % a hack to toggle sections
\usepackage{ifthen}
\usepackage{mdframed}
\usepackage{verbatim}
\usepackage[strings]{underscore}
\usepackage{listings}
\usepackage{bbm}
\rhead{}
\lhead{}

\renewcommand{\baselinestretch}{1.15}

% Shortcuts for commonly used operators
\newcommand{\E}{\mathbb{E}}
\newcommand{\Var}{\operatorname{Var}}
\newcommand{\Cov}{\operatorname{Cov}}
\newcommand{\Bias}{\operatorname{Bias}}
\DeclareMathOperator{\argmin}{arg\,min}
\DeclareMathOperator{\argmax}{arg\,max}

% do not number subsection and below
\setcounter{secnumdepth}{1}

% custom format subsection
\titleformat*{\subsection}{\large\bfseries}

% set up the \question shortcut
\newcounter{question}[section]
\newenvironment{question}[1][]
  {\refstepcounter{question}\par\addvspace{1em}\textbf{Question~\Alph{question}\!
    \ifthenelse{\equal{#1}{}}{}{ [#1 points]}: }}
    {\par\vspace{\baselineskip}}

\newcounter{subquestion}[question]
\newenvironment{subquestion}[1][]
  {\refstepcounter{subquestion}\par\medskip\textbf{\roman{subquestion}.\!
    \ifthenelse{\equal{#1}{}}{}{ [#1 points]:}} }
  {\par\addvspace{\baselineskip}}

\titlespacing\section{0pt}{12pt plus 2pt minus 2pt}{0pt plus 2pt minus 2pt}
\titlespacing\subsection{0pt}{12pt plus 4pt minus 2pt}{0pt plus 2pt minus 2pt}
\titlespacing\subsubsection{0pt}{12pt plus 4pt minus 2pt}{0pt plus 2pt minus 2pt}


\newenvironment{hint}[1][]
  {\begin{em}\textbf{Hint: }}{\end{em}}

\ifshowsolutions
  \newenvironment{solution}[1][]
    {\par\medskip \begin{mdframed}\textbf{Solution~\Alph{question}#1:} \begin{em}}
    {\end{em}\medskip\end{mdframed}\medskip}
  \newenvironment{subsolution}[1][]
    {\par\medskip \begin{mdframed}\textbf{Solution~\Alph{question}#1.\roman{subquestion}:} \begin{em}}
    {\end{em}\medskip\end{mdframed}\medskip}
\else
  \excludecomment{solution}
  \excludecomment{subsolution}
\fi

\newcommand{\boldline}[1]{\underline{\textbf{#1}}}

\chead{%
  {\vbox{%
      \vspace{2mm}
      \large
      Machine Learning \& Data Mining \hfill
      Caltech CS/CNS/EE 155 \hfill \\[1pt]
      Miniproject 2\hfill
      Released February $17^{th}$, 2017 \\
    }
  }
}

\begin{document}
\pagestyle{fancy}

% LaTeX is simple if you have a good template to work with! To use this document, simply fill in your text where we have indicated. To write mathematical notation in a fancy style, just write the notation inside enclosing $dollar signs$.

% For example:
% $y = x^2 + 2x + 1$

% For help with LaTeX, please feel free to see a TA!

%%%%%%%%%%%%%%%%%%%%%%%%%%%%%%%%%%%%%%%%%%%%%%%%%%%%%%%%%%%%%%%%%%%%%%%%%%%%%%%%%%%%%%%%

\section{Introduction}
\medskip
\begin{itemize}

    \item \boldline{Group members:} Enrico Borba, Claire Goeckner-Wald
    \item \boldline{Team name:} Papa Mart's Mini Gary - The Comeback
    \item \boldline{Division of labour:}
        Enrico Borba: Programming, ideas, report visualization.
        Claire Goeckner-Wald: Programming, ideas, report assembly.

\end{itemize}

%%%%%%%%%%%%%%%%%%%%%%%%%%%%%%%%%%%%%%%%%%%%%%%%%%%%%%%%%%%%%%%%%%%%%%%%%%%%%%%%%%%%%%%%

\section{Pre-processing}
\medskip
\begin{itemize}
    % Explain your choices, as well why you chose these choices initially. What was your final pre-processing? How did you tokenize your words, and split up the data into separate sequences? What changed as you continued on your project? What did you try that didn’t work? Also write about any analysis you did on the dataset to help you make these decisions.

    %%%%%%%%%%%%%%%%%%%%%%
    \item \boldline{Topic}
    \begin{itemize}
        \item \textbf{Subtopic:}
        \item \textbf{Subtopic:}
    \end{itemize}


\end{itemize}

%%%%%%%%%%%%%%%%%%%%%%%%%%%%%%%%%%%%%%%%%%%%%%%%%%%%%%%%%%%%%%%%%%%%%%%%%%%%%%%%%%%%%%%%

\section{Unsupervised Learning}
\medskip
\begin{itemize}
    % Your report should also contain a section highlighting your HMM. What packages did you use, if any? How did you choose the number of hidden states?

    %%%%%%%%%%%%%%%%%%%%%%
    \item \boldline{Topic}

    \begin{itemize}
    \item \textbf{Subtopic:}
    \item \textbf{Subtopic:}
    \end{itemize}

    %%%%%%%%%%%%%%%%%%%%%%
    \item \boldline{Topic}

    \begin{itemize}
    \item \textbf{Subtopic:}
    \item \textbf{Subtopic:}
    \end{itemize}

    % If you would like to insert a figure, you can just use the following five lines, replacing the image path with your own and the caption with a 1-2 sentence description of what the image is and how it is relevant or useful.
    % \begin{figure}[H]
    % \centering
    % \includegraphics[width=\textwidth]{smiley.png}
    % \caption{Insert caption here.}
    % \end{figure}


\end{itemize}

%%%%%%%%%%%%%%%%%%%%%%%%%%%%%%%%%%%%%%%%%%%%%%%%%%%%%%%%%%%%%%%%%%%%%%%%%%%%%%%%%%%%%%%%

\section{Visualization \& Interpretation}
\medskip
\begin{itemize}
    % In your report, you should explain your interpretation of how a Hidden Markov Model learns patterns in Shakespeare’s texts. You should briefly elaborate on the methods you used to analyze the model. In addition, for at least 5 hidden states give a list of the top 10 words that associate with this hidden state and state any common features these groups. Furthermore, try to interpret and visualize the learned transitions between states. A possible suggestion is to draw a transition diagram of your markov model and give de- scriptive names to the sates. Feel free to be creative with your visualizations, but remember that accurately representing data is still your primary objective. Your figures, tables, and diagrams should contribute to a discussion about your model.

    %%%%%%%%%%%%%%%%%%%%%%
    \item \boldline{Topic}

    \begin{itemize}
    \item \textbf{Subtopic:}
    \item \textbf{Subtopic:}
    \end{itemize}

\end{itemize}

%%%%%%%%%%%%%%%%%%%%%%%%%%%%%%%%%%%%%%%%%%%%%%%%%%%%%%%%%%%%%%%%%%%%%%%%%%%%%%%%%%%%%%%%

\section{Poetry Generation}
\medskip
\begin{itemize}
    % In your report, describe your algorithm for generating the 14 line sonnet. Include at least one sonnet generated from your unsupervized trained HMM in your final report as an example. You should comment on the quality of geneating poems in this naive manner. How is the accurate is rhyme, rythym, and syllable count to what a sonnet should be? Do your poems make any sense? Does it retain Shakespeare’s original voice? How does training with different number of hidden states effect the poems generated (in a qualita- tive manner)? For the good qualities that you describe, also discuss how you think the HMM was able to capture these qualities.

    %%%%%%%%%%%%%%%%%%%%%%
    \item \boldline{Topic}

    \begin{itemize}
    \item \textbf{Subtopic:}
    \item \textbf{Subtopic:}
    \end{itemize}


\end{itemize}







%%%%%%%%%%%%%%%%%%%%%%%%%%%%%%%%%%%%%%%%%%%%%%%%%%%%%%%%%%%%%%%%%%%%%%%%%%%%%%%%%%%%%%%%

\section{Additional Goals}
\medskip
\begin{itemize}
    % Talk about the extra improvements you made to your poem generation algorithm. What was the prob- lems you were trying to fix? How did you go about attempting to fix it? Why did you think that what you tried would work? Did your method succeed in making the sonnet more like a sonnet? If not, why do you think what you tried didn’t work? What tradeoffs do you see in quality and creativity when you make these changes?

    %%%%%%%%%%%%%%%%%%%%%%
    \item \boldline{Topic}

    \begin{itemize}
    \item \textbf{Subtopic:}
    \item \textbf{Subtopic:}
    \end{itemize}

\end{itemize}

%%%%%%%%%%%%%%%%%%%%%%%%%%%%%%%%%%%%%%%%%%%%%%%%%%%%%%%%%%%%%%%%%%%%%%%%%%%%%%%%%%%%%%%%

\section{Extra Credit: Recurrent Neural Networks}
\medskip
\begin{itemize}
    %Explain in detail what model you implemented using what packages. What parameters did you tune? How does an RNN/LSTM compare in poem quality to the HMM? How does it compare in runtime/amount of training data needed to the HMM? Include a poem that you generated using your recurrent model.

    %%%%%%%%%%%%%%%%%%%%%%
    \item \boldline{Topic}

    \begin{itemize}
    \item \textbf{Subtopic:}
    \item \textbf{Subtopic:}
    \end{itemize}

\end{itemize}

%%%%%%%%%%%%%%%%%%%%%%%%%%%%%%%%%%%%%%%%%%%%%%%%%%%%%%%%%%%%%%%%%%%%%%%%%%%%%%%%%%%%%%%%

\end{document}